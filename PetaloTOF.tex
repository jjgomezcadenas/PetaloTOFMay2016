\pdfoutput=1

\documentclass[11pt,a4paper]{article}
\usepackage{jinstpub}

\let\ifpdf\relax

\bibliographystyle{ieeetr}

\input{Commands}

%%% First page %%%
\title{Investigation of the Coincidence Resolving Time performance of a PET scanner based on liquid xenon: A Monte Carlo study}
\author[a]{J.J.~Gomez-Cadenas,}
\author[a]{J.M.~Benlloch-Rodr\'iguez,}
\author[a,1]{P.~Ferrario,}\note{Corresponding author.}
\emailAdd{paola.ferrario@ific.uv.es}
\author[b]{F.~Monrabal,}
\author[a]{J.~Rodr\'iguez,}
\author[c]{J.F. Toledo}
\affiliation[a]{Instituto de F\'isica Corpuscular (IFIC), CSIC \& Universitat de Val\`encia,\\
Calle Catedr\'atico Jos\'e Beltr\'an, 2, 46980 Paterna, Valencia, Spain}
\affiliation[b]{Department of Physics, University of Texas at Arlington\\
Arlington, Texas 76019, USA}
\affiliation[c]{Instituto de Instrumentaci\'on para Imagen Molecular (I3M), Universitat Polit\`ecnica de Val\`encia\\ 
Camino de Vera, s/n, Edificio 8B, 46022 Valencia, Spain}


\abstract{The measurement of the time of flight of the two 511 keV gammas recorded in coincidence in a PET scanner provides an effective way of reducing the random background and therefore increases the scanner sensitivity, provided that the coincidence resolving time (CRT) of the gammas is sufficiently good. The best commercial PET-TOF system today (based in LYSO crystals and digital SiPMs), is the VEREOS of Philips, boasting a CRT of 316 ~ps (FWHM). 
%The CRT attained in small setups, however, is much better (around 100-200 ps, depending on the experimental conditions). 

%PETALO (Positron Electron TOF Apparatus using Liquid xenOn) is a proposed new type of PET scanner using liquid xenon. The detector building block is the liquid xenon scintillating cell. The cell shape and dimensions can be optimised depending on the intended application. In particular, the LXSC2 instruments the entry and exit faces of the box (relative to the gammas line of flight) with silicon photomultipliers, while all the other faces are covered with a reflecting material such as Teflon. The SiPMs can be VUV sensitive or, alternatively, be coated with a wavelength shifter such as tetraphenyl butadiene. The good energy and position resolution of the LXSC2 coupled with the relatively low cost of xenon (a factor 7-8 less than LYSO per unit of detection) makes PETALO suitable for the construction of a large scanner of high sensitivity.   

In this paper we present a Monte Carlo investigation of the CRT performance of a
PET scanner exploiting the scintillating properties of liquid xenon. We find that an excellent CRT of 60--70 ps (depending on the PDE of the sensor) can be obtained if the scanner is instrumented with silicon photomultipliers (SiPMs) sensitive to the ultraviolet light emitted by xenon. Alternatively, a CRT of 120 ps can be obtained instrumenting the scanner with (much cheaper) blue-sensitive SiPMs coated with a suitable wavelength shifter.
These results show the excellent time of flight capabilities of a PET device based in liquid xenon.   
}

\keywords{PET, TOF, liquid xenon, energy resolution,
 high sensitivity, coincidence resolving time (CRT), SiPMs, LYSO}

\begin{document}

\maketitle

\section{Introduction}
Positron Emission Tomography (PET) is a non invasive imaging technique that produces a three-dimensional image of functional processes --it does not show anatomic features, but it rather  measures the metabolic activity of the cells-- in the body. PET is used in
both clinical and pre-clinical research to study the molecular bases and treatments of
disease. The principle of operation relies in injecting into the patient a  
biologically active molecule doped with a radioactive isotope, called tracer (a standard tracer is fluorodeoxyglucose, formed substituting an atom of oxygen by the isotope \FDG\ in a glucose molecule). The radionuclide decays and the resulting positrons
subsequently annihilate with electrons after traveling a short distance within the subject.
Each annihilation produces two 511 keV photons emitted most of the time in opposite
directions and these photons are registered by a detection system. For a pair coincidence event, if the energy of two photons stays within a
pre-set energy window, centred on the 511 keV photopeak, and the timing difference stays within a pre-set time window, a coincidence event will be registered and constitutes a line-of-response (LOR) for image
reconstruction.

The reconstruction of the image in a PET system requires crossing many LORs which in turn define one emission point in the area under study. A LOR can be formed by: 1) a {\bf true coincidence}, which occurs when both photons from an annihilation event are detected, neither photon undergoes any form of interaction prior to detection, and no other event is detected within the coincidence time-window; 2)
a {\bf scattered coincidence}, which occurs when one or both detected photons undergo at least one Compton scattering interaction prior to detection;
%Since the direction of the photon changes due to the scattering process, the resulting coincidence event will, most likely, produce a wrong LOR. 
3)  a {\bf random coincidence}, which occurs when two photons, not arising from the same annihilation event, impinge the detectors within the coincidence time window of the system. Scatter and random coincidences add a background to the true coincidence distribution, decreasing contrast and causing the isotope concentrations to be overestimated. They also add statistical noise to the signal. 

The measurement of the time of flight (TOF) of the two 511 keV gammas recorded in coincidence in a PET scanner provides an effective way of reducing the background due to random coincidences and therefore increases the scanner sensitivity, provided that the coincidence resolving time (CRT) of the gammas is sufficiently good. Existing commercial systems based in LYSO crystals, such as the GEMINI \cite{gemini}, and most recently the 
VEREOS \cite{vereos} (both of them manufactured by Philips) reach CRT values of $\sim$ 600 (300) ps (FWHM). 

Recent research, using small setups, has found even better results. For example Ferri et al. \cite{LysoFBK} report measurements using a pair of small LYSO crystals 
($3 \times 3 \times 5 {\rm\ mm^3}$ )  read out by high-fill factor SiPMs. With this setup, a CRT of $157\pm 5$ ps was found at ambient temperature ($20\,^{\circ}{\rm C}$). The CRT was considerably improved working at $-20\,^{\circ}{\rm C}$, to a value of
 $125\pm 5$ ps. The limiting factor of the CRT at ambient temperature was the impact of the dark current rate (DCR). Gundacker and co-workers have found a CRT of $85 \pm 4 $ ps for very small crystals of $2 \times 2 \times 3 {\rm\ mm^3}$  \cite{LysoCRT}.

In this paper we present a Monte Carlo investigation of the CRT performance of a recently proposed new type of
PET scanner, called PETALO (Positron Electron TOF Apparatus using Liquid xenOn). 
 In section \ref{sec.LXe} we summarize the relevant properties of liquid xenon (LXe) as scintillating material, review  the mechanisms which produce scintillation light in LXe and propose a parameterization of the measured scintillation data in terms of two decay constants. In section \ref{sec.PET} we present the main features of the PETALO scanner. In section \ref{sec.CRT}  we discuss the CRT performance in LXe cells using SiPMs sensitive to the xenon ultraviolet  light and conventional SiPMs coated with a suitable wavelength shifter. Conclusions are presented in section \ref{sec.conclu}.  
 %In section \ref{sec.scint} we review the mechanisms which produce scintillation light in LXe and propose a parameterization of the measured scintillation data in terms of two decay constants. and the main features of the PETALO scanner

%
%In this paper we present a Monte Carlo investigation of the time resolution
%that could be achieved by a PETALO-LXSC2 scanner. To systematically study the different factors that enter the CRT, we have simulated a LXe setup and a LYSO setup, used as a reference. The LXe setup consists of two LXSC2 cells of 
%$2.4 \times 2.4 \times 5 {\rm ~cm^3}$, facing each other in opposite sides of a 511 keV gamma source 
%(figure \ref{fig.psetup}) and equipped with SiPMs which can be either VUV sensitive or coated with  TPB. The  LYSO setup replaces the two LXSC2 cells by a pair of  monolithic LYSO crystals, also readout by SiPMs, of identical dimensions.
%of identical transverse dimensions ($2.4 \times 2.4 {\rm ~cm^2}$) and identical thickness in terms of interaction lengths ($1.7 {\rm cm}$).  


\section{Liquid xenon as detection material}
\label{sec.LXe}


\subsection*{Physical properties of liquid xenon relevant for PET}


\begin{figure}[!bhtp]
	\centering
	\includegraphics[scale=0.6]{img/ScintillationSpectrumLXe.pdf}
	\caption{\label{fig.spectrumLXe} Emission spectrum of scintillation photons in LXe, simulated as a gaussian distribution with mean 178 nm and FWHM 14 nm \cite{EmissionLXe}.}
\end{figure}

Xenon is a noble gas. It responds to the interaction of ionizing radiation producing about 60 photons per keV of deposited energy \cite{Chepel:2012sj}. The emitted photons have wavelengths in the ultraviolet range 
(figure \ref{fig.spectrumLXe}) with an average wavelength of 178 nm. The scintillation signal is fast 
(see section \ref{sec.scint}) and can thus result in an excellent CRT.  In its liquid phase (the triple point of xenon is reached at a temperature of 161.35 K and a pressure of 0.816 bar \cite{TempLXe}) xenon has a reasonable high density (2.98 g/cm$^3$)  \cite{DensityLXe}, which gives an attenuation length of 3.7 cm for 511-keV gammas \cite{AttLengthLXe}  and an acceptable light attenuation length (36.4 cm) \cite{RayleighLXe}, which makes it suitable for PET applications. Its main attractive features are:

\begin{enumerate}
\item {\bf A high scintillation yield ($\sim$ 30\,000 photons per 511 keV gamma)}, twice as large as that of LYSO. 
\item {\bf LXe is a continuous medium with uniform response}. Therefore, the design of a compact system is much simpler than in the case of solid detectors of fixed shape. It is also possible to provide a 3D measurement of the interaction point, and, thus, a high resolution measurement of the depth of interaction (DOI). Furthermore, in LXe it is possible to identify Compton events depositing all their energy in the detector as separate-site interaction, due to its relatively large interaction length.  
Once an event in the region of interest (around 511 keV of total deposited energy) is identified, the pattern of recorded light on SiPMs can in principle be inspected to find one or more depositions, using, for instance, neural networks \cite{tfm}.This increases the sensitivity of the system, since those events can, in principle, be used for image reconstruction. 
\item {\bf The temperature at which xenon can be liquefied at a pressure very close to the atmospheric} is high enough as to be reached using a basic cryostat. Also, at this temperature SiPMs can be operated normally, and their DCR is reduced by a factor $\sim 2^{13}$, thus making it essentially negligible. 
\item {\bf The cost} of LXe is 3 \$/cc to be compared with $\sim$ 40--50 \$/cc in the case of LYSO. 
 \end{enumerate}

\subsection*{Scintillation in LXe} \label{sec.scint}

When a 511 keV gamma interacts in liquid xenon it will produce a secondary electron (by either photoelectric, which happens in $\sim$ 20 \% of the events, or Compton interaction) which in turn will propagate a short distance in the liquid, ionizing the medium. Most of the scintillation light in xenon is emitted by diatomic excited molecules which are formed through two distinct processes:
\begin{enumerate}
\item Excitation of atoms by electron impact with subsequent formation of strongly bound diatomic molecules in the excited state (excitons). We refer to this process as scintillation due to exciton self-trapping or SE.
\item Recombination of the ionization electrons with positive ions. We call this process scintillation due to recombination or SR. 
\end{enumerate}

Both processes are very fast (of the order of the ps) and therefore scintillation in LXe does not have a physical rise constant. The two decay constants are related with the decay of the two lowest excited molecular states 
(${}^1{\Sigma_u^+},{}^3{\Sigma_u^+}$). The decay of the singlet state 
(${}^1{\Sigma_u^+}$) occurs with a lifetime $\tau_1 =2.2$ ns, while the lifetime of the triplet state (${}^3{\Sigma_u^+}$) is $\tau_2 =27$ ns \cite{Kubota79}.

The relative contribution of SE and SR processes to the LXe scintillation light has been measured to be:
\begin{equation}
I = 0.3\times I^S + 0.7\times I^R
\label{eq.tot}
\end{equation}
%
where $I^S$~refers to the intensity of the SE process and $I^R$ to the intensity of the SR process.

Also, after the measurements in reference \cite{Kubota79}, $I^S$ and $I^R$ can be written as:
\begin{eqnarray}
I^S & = & 0.15\times I_1^S(t) +  0.85\times I_2^S(t) \\
I^R & = & 0.44\times I_1^R(t) +  0.56\times I_2^R(t)
\label{eq.1}
\end{eqnarray}

The time dependence of the SE process is described by:
\begin{equation}
I^S_i = \frac{e^{-t/\tau_i}}{\tau_i} 
\label{eq.2}
\end{equation}
%
where $i=1,2$. The time dependence of the RE process is more complicated, since the recombination time $T_r$
of the electrons is slow compared with the fast constant $\tau_1$. Then, $I^R_1$~can be parametrized as: 

\begin{equation}
I^R_1 =(1 + \frac{t}{T_r})^{-2}
 \label{eq.3}
\end{equation}
%
where $T_r = 15$ ns for xenon. On the other hand, since $\tau_2 > T_r$, $I^R_2$ can be described as in the case of the SE processes: 

\begin{equation}
I^R_2 =\frac{e^{-t/\tau_2}}{\tau_2}
 \label{eq.4}
\end{equation}

\begin{figure}[!bhtp]
	\centering
	\includegraphics[scale=0.6]{img/Fit10ns_ROOT.pdf}
%	\includegraphics[width=.5\textwidth]{../img/TOFPET.jpg}
	\caption{\label{fig.scint} Intensity of scintillation in LXe as a function of time. Blue dots are obtained from equations \ref{eq.tot} to \ref{eq.4}, using time constants and intensities from reference \cite{Kubota79}. The red line is the same data fitted with the function in equation \ref{eq.scint}. Although the distribution extends up to several hundreds of nanoseconds, only the first 10 ns are shown, which is the range used for the fit. In the simulation studies presented in this work, only the first nanoseconds are relevant, since the CRT is determined by the arrival time of first few photoelectrons.}
\end{figure}

Figure \ref{fig.scint} shows the total intensity of scintillation in LXe for a 511 keV gamma (resulting in a total of $N_\gamma =30\,000$ VUV photons), adding the contribution of SE and SR processes (blue dots). This distribution is well described in terms of the sum of two exponential distributions with decay constants $\tau_1$ and $\tau_2$. The fit, shown in figure \ref{fig.scint}, gives the following distribution for the number of photons: 

\begin{equation}
I(t) = N_\gamma (0.07 \times \frac{e^{-t/\tau_1}}{\tau_1} + 0.93 \times \frac{e^{-t/\tau_2}}{\tau_2})
\label{eq.scint}
\end{equation}
%
Equation \ref{eq.scint} fits better in the short range, which is the most relevant for CRT estimations. This parametrization is the one used in the simulation studies throughout this paper, since the Geant-4 simulation toolkit on which our simulation is based \cite{Agostinelli:2002hh} can model the production of scintillation photons as a single or double exponential.

%Using equation \ref{eq.scint} it is possible to estimate the potential of LXe for TOF measurements, in particular compared with LYSO. The intrinsic CRT of a scintillator is proportional to the figure of merit 
%$\psi = N_{pe}/\tau$, where $\tau$~is the lifetime of the scintillator and $N_{pe}$~is the number of photoelectrons, which in turn can be written as $N_{pe} = N_\gamma \times $PDE. Here $N_\gamma$ is the total number of photons emitted by the scintillator and PDE the overall particle detection efficiency of the sensor reading the scintillation light. 
%
%Then, for LYSO: 
%
%\begin{equation}
%\psi = \frac{14000}{40} \times PDE = 350 \times PDE
%\label{eq.lysoScint}
%\end{equation}
%%
%while, for LXe:
%
%\begin{eqnarray}
%\psi_1 &=& \frac{30000 \times 0.07}{2.2} \times PDE = 954 \times PDE \\
%\psi_2 &=& \frac{30000 \times 0.93}{27} \times PDE = 1033 \times PDE \\
%\end{eqnarray}
%
%Therefore:
%
%\begin{equation}
%\frac{\psi_{LXe}}{\psi_{LYSO}} = \frac{1987}{350} \sim 6 \times PDE
%\label{eq.ratioScint}
%\end{equation}
%
%On the other hand, the scintillation light in LYSO is blue, peaking around 440 nm where the PDE of today's SiPMs reaches values as high as 50\%. Instead, the scintillation light of LXe peaks around 178 nm (VUV region). The NEXT experiment, for instance, reads the LXe VUV scintillation light using conventional SiPMs coated with TPB \cite{Alvarez:2013gxa}. Alternatively, the MEG experiment will upgrade its LXe calorimeter using VUV-sensitive SiPMs \cite{Ogawa:2015ucj}.
%The devices tested for the MEG upgrade have a PDE of about 15\%, and reaching a PDE of 20\% seems possible. In the case of SiPMs coated with TPB, the VUV light is absorbed by the wavelength shifter (with an efficiency of 80\%) and then re-emitted isotropically. Thus, the effective PDE is reduced to about 20\%. Thus:
%
% \begin{equation}
%\frac{\psi_{LXe}}{\psi_{LYSO}} = \frac{1987}{350} \sim 3
%\label{eq.ratioScint}
%\end{equation}
%
%In practice, the figure of merit is insufficient to estimate accurately the CRT resolution, since one has to add many other experimental effects. In the next section we address the most important ones through a detailed Monte Carlo study. 
%


\section{PETALO}\label{sec.PET}

\subsection*{The PETALO concept}

\begin{figure}[!htbp]
	\centering
	\includegraphics[scale=0.20]{img/Box_2faces_1.jpg}
	\includegraphics[scale=0.20758]{img/Box_2faces_3.jpg}
	\caption{The LXSC2 instruments the entry and exit faces of the box (relative to the photon line of flight) with silicon photomultipliers (SiPMs), which can be eventually coated with TPB. The non-instrumented faces are covered by reflecting Teflon (optionally also coated with TPB).}\label{fig.box} 
\end{figure}

%The first idea of using a LXe time projection chamber for PET was proposed in 1993 by Chepel \cite{chepelFirst}. 
The possibility of building a LXe PET based on the excellent properties of LXe as scintillator was first suggested by Lavoie in 1976 \cite{lavoie}, and the study of this type of PET was carried out by the Waseda group \cite{Doke1,Nishikido2,Nishikido1}. The Waseda prototype was based in LXe cells read out by VUV-sensitive PMTs which covered 5 of the 6 sides of the cell. In 1993, Chepel proposed to read both scintillation and ionization charge in a LXe Time Projection Chamber \cite{chepelFirst} and an extensive work has been done since then by him and co-workers \cite{chepelLXe, chepelRes, chepelEnergyRes}.

PETALO (Positron Electron TOF Apparatus using Liquid xenOn) is a proposed new type of PET scanner which exploits the copious scintillation of liquid xenon and the availability of state-of-the-art SiPMs and fast electronics designed to maximize TOF performance \cite{Petalo2015}. The detector building block is the liquid xenon scintillating cell (LXSC). The cell shape and dimensions can be optimised depending on the intended application. In particular, the LXSC2 instruments the entry and exit faces of the box (relative to the gammas line of flight) with silicon photomultipliers, while all the other faces are covered with a reflecting material such as Teflon (figure \ref{fig.box}). They are read out by ASICs optimized for excellent timing resolution. PETALO is a compact, homogenous and highly efficient detector which shares many of the desirable properties of monolithic crystals, with the added advantage of high yield and fast scintillation offered by liquid xenon, low noise due to cryogenic operations which virtually eliminates the SiPMs DCR, and the potential of low cost. 

The energy resolution $R$ of a liquid xenon scintillation detector is a combination of light collection variation due to the detector geometry $R_g$, the statistical fluctuation of number of photoelectrons from the sensors $R_s$, the fluctuation of electron-ion recombination due to escape electrons $R_r$, and the intrinsic resolution from liquid xenon scintillation light $R_i$, due to the non-proportionality of scintillation yield, also associated to fluctuations in the number of secondary electrons. Therefore:

\begin{equation}
R^2 = R_g^2 + R_s^2 + R_r^2 + R_i^2
\end{equation}

The contribution of the intrinsic terms, ($R_r$~and $R_i$) has been measured to be 11 \% FWHM \cite{aprileRes} for 511 keV gammas. The contribution of photoelectron statistics and geometrical effects for the LXSC2 configuration was found to be 3 \% FWHM \cite{tfm}. The combination of both effects yields an expected 
overall resolution of the order of 12\% FWHM for 511 keV gammas, similar to that of LYSO detectors. 

%% Referee's concern about the digital resolution
%The spatial resolution in the (x,y) coordinates (transverse to the gammas line of flight) is obtained in the LXSC2 by the weighted pulses in the SiPMs. The digital resolution would be $p/\sqrt{12}$, where $p$ is the pitch. With a pitch of 6 mm a 1.7 mm rms or 4 mm FWHM could be expected. Weighting with the SiPM amplitude improves the resolution to about 2 mm FWHM. The DOI is obtained by computing the ratio between the total signal recorded in the entry and exit face, and its resolution is found to be also 2 mm FWHM. 
The spatial resolution in the (x,y) coordinates (transverse to the gammas line of flight), meaning the precision in identifying the position of a single-vertex interaction, is obtained in the LXSC2 by the weighted pulses in the SiPMs. Weighting with the SiPM amplitude using a centre-of-gravity algorithm, gives a resolution of about 2 mm FWHM \cite{tfm}. The DOI is obtained by computing the ratio between the total signal recorded in the entry and exit face, and its resolution is found to be also 2 mm FWHM \cite{tfm}. 

\subsection*{VUV-sensitive SiPMs versus SiPMs coated with TPB}


%\begin{figure}[!htbp]
%	\centering
%	\includegraphics[scale=0.8]{img/PDEVUV.png}
%	\caption{Photo detection efficiency (PDE) of different VUV sensitive SiPMs
%	(from reference \cite{vuv}). The MEG devices are
%	manufactured by Hamamatsu Photonics, while the FBK-2010 and the RGB-HD
%	are manufactured by FBK.  }\label{fig.vuv} 
%\end{figure}

%\begin{figure}[!htbp]
%	\centering
%	\includegraphics[scale=0.9]{img/TPBEfficiency.png}
%	\caption{Total integrated fluorescence efficiency for a thin layer of TPB, as a function of input  photon wavelength.
%	Figure from reference \cite{Gehman:2011xm}.  }\label{fig.tpbeff} 
%\end{figure}
%
\begin{figure}[!htbp]
	\centering
	\includegraphics[scale=0.9]{img/TPBSpectrum.png}
	\caption{Visible re-emission spectrum for a TPB film illuminated with 
	128, 160, 175, and 250 nm light. All spectra are normalized to unit area.
	Figure from reference \cite{Gehman:2011xm}.  }\label{fig.tpb} 
\end{figure}


%
%\begin{figure}[!htbp]
%	\centering
%	\includegraphics[scale=0.6]{img/TPBdecays.pdf}
%	\caption{Evolution of the decay constant of TPB as a function of the thickness of the TPB layer compared
%	with the thickness of the substrate (a thin quartz film). The TPB decay constant decreases as the thickness of the TPB layer (thus its concentration) increases, reaching an asymptotic value of 2.2 ns. Figure elaborated from reference \cite{TPBtau}.  }\label{fig.tpbtau} 
%\end{figure}

The scintillation light of LXe peaks around 178 nm (VUV region). Therefore the LXSC2 needs to be instrumented either with VUV-sensitive SiPMs (as for example the upgraded LXe calorimeter of the 
MEG experiment \cite{Ogawa:2015ucj}) or by conventional SiPMs coated with a wavelength shifter such as tetraphenyl butadiene (TPB) (as for example in the NEXT experiment \cite{Alvarez:2013gxa}). 
The VUV-SiPMs tested for the MEG as well as for the future nEXO 
experiment \cite{Ogawa:2015ucj,Ostrovskiy:2015oja} reach a PDE of about 
15\%. Further improvements of the technology could eventually result in a PDE of 20\%.  Conventional SiPMs can reach today a PDE of around 50\% \cite{pde}. When they are coated with a thin layer of TPB, the VUV light is absorbed by the wavelength shifter with an efficiency of 80\%,  
%(figure \ref{fig.tpbeff}) 
 %(figure \ref{fig.vuv})
shifted to $\sim$ 430 nm 
(figure \ref{fig.tpb}) 
and re-emitted isotropically \cite{Gehman:2011xm}. The decay constant of TPB has been measured  in thin films \cite{TPBtau} and a value of 2.2--3 ns is found, depending on the TPB concentration. %
%An important application of the PETALO concept would be the construction of a ``full body PET'', of axial dimensions in the range of 70-100 cm, exploiting the low cost and the good performance offered by LXe. In the rest of this paper we show that the excellent CRT that can be achieved by the LXSC2 implies that a PETALO scanner could also deploy much better TOF performance that that achieved by current  commercial devices.   

\section{Monte Carlo study of the CRT in the LXSC2} 
\label{sec.CRT}

To systematically study the different factors that enter the CRT, we have simulated a LXe setup and a LYSO setup, using the Geant-4 toolkit \cite{Agostinelli:2002hh}.  All the simulations described in this paper use this set-up. The LXe setup consists of two LXSC2 of 
$2.4 \times 2.4 \times 5 {\rm ~cm^3}$, facing each other in opposite sides of a 511 keV gamma source 
(figure \ref{fig.psetup}) and equipped with SiPMs which can be either sensitive to the xenon ultraviolet (VUV) light, or coated with a wavelength shifter such as tetraphenyl butadiene (TPB). The  LYSO setup replaces the two LXe cells by a pair of  monolithic LYSO crystals of identical dimensions, also read out by SiPMs.
In both configurations, the uninstrumented faces reflect optical photons according to a lambertian distribution with an efficiency of 97 \%. Rayleigh scattering is simulated, with an attenuation length of 36 cm.

 \begin{figure}[!bthp]
	\centering
	\includegraphics[scale=0.4]{img/PetaloSetupAxes.png}
	\caption{\label{fig.psetup} Monte Carlo simulation of gamma interactions in the LXe setup, as 
	described in the text. Gammas are emitted from (0, 0, 0) in direction $(0.0363, -0.0644, -0.9972)$ and its opposite one. }
\end{figure}

%% Polar coordinates: theta = 0.0703112, phi = -1.10715, r = 127.315 mm

%
%\begin{figure}[!bhtp]
%	\centering
%	\includegraphics[scale=0.5]{../img/TOFSetup.png}
%	\caption{The setup simulated by this study. Back to back 511 keV gammas are emitted
%	isotropically from $V_0$, and interact in two identical LXSC2, at vertices $V_1$~and
%	$V_2$. The CRT ($\Delta t$), is found by subtracting the time of the first photoelectron recorded
%	in each LXSC, $t_1$~and $t_2$~(or taking an average, see text for details) and correcting
%	by the time of flight of the gammas ($\Delta d_g/c$) and the time of flight of the scintillation photons
%	($n \cdot \Delta d_p/c$). }\label{fig.setup} 
%\end{figure}  

  Back to back 511 keV gammas are shot isotropically, from a common vertex within the solid angle covered by two opposite LXSC2 ($2.4 \times 2.4 \times 5\ {\rm cm^3}$). The gammas interact in the LXSCs producing scintillation photons according to equation \ref{eq.scint} for LXe (in the case of LYSO they are produced according to a single decay constant of 40 ns) and propagated through the material. The photons reach eventually the SiPMs. If the SiPM is not coated with TPB, the photon will enter the SiPM window if the angle of incidence is below the critical angle defined by the ratio of the refraction indexes $n_2$, $n_1$ of the LXSC (LYSO) and the SiPM window $\sin \theta_c = n_2/n_1$. SiPMs are usually covered with a passivation layer, whose refractive index is similar to that of SiO$_2$ and is around 1.4 --1.5 \cite{nSiPM}.
  %,  where we take $n_1$ = 1.54 as the refraction index of the SiPM window and�  $n_2$ = 1.7 (1.8) as the refraction index of the LXe (LYSO).
   If the SiPM is coated with TPB, the 
  VUV photon is absorbed with an efficiency of 80\% and re-emitted isotropically with a wavelength given by figure \ref{fig.tpb}, then enters the SiPM window with a much larger critical angle, since  $n_2 \sim n_1$~for LXe in the visible region. Depending on the PDE of the SiPM the incident photon may result in a photoelectron.
We denote $d_g$ as the distance from the centre of the LOR to the interaction vertex of each 511 keV gamma, $d_d$ as the displacement of the gamma emission vertex from the centre of the LOR and $d_p$ as the distance from the interaction vertex to the detection vertex (i.e., the position of the sensor). If $t_1,t_2$ are the time of the first photoelectron recorded in each one of the LXSC, the time difference between them can be written as:
%
\begin{equation}
t_1 - t_2 = 2\frac{d_d}{c} + \frac{\Delta d_g}{c} + \frac{ \Delta d_p}{v_p}
\end{equation}
where $v_p$ is the velocity of the scintillating photon. Therefore, the difference in time between the gamma emission vertex and the centre of the LOR can be expressed as:
%
\begin{equation}
\Delta t  \equiv \frac{d_d}{c}  =  \frac{1}{2}(t_1 - t_2 - \frac{\Delta d_g}{c} - \frac{\Delta d_p}{v_p}) 
\label{eq.CRT}
\end{equation}
%
%   Denoting $t_1,t_2$ as the time of the first photoelectron recorded in each one of the LXSC, the time difference between them can be written as:
%
%\begin{equation}
%\Delta t = \frac{1}{2}(t_2 - t_1 - \frac{\Delta d_g}{c} - \frac{n \cdot \Delta d_p}{c}) 
%\label{eq.CRT}
%\end{equation}
%%
%where $\Delta d_g/c$~is the time of flight difference from the emission to the interaction vertex of the two 511 keV gammas, and $n \cdot \Delta d_p/c$~is the time of flight difference from the interaction vertex to the
%detection vertex (i.e., the position of the sensor) of the scintillation photons. 
The CRT,
$\delta \Delta t$, is defined as the variance\footnote{expressed in FWHM unless explicitly stated otherwise.} of the $\Delta t$ distribution. 

\begin{figure}[!bhtp]
	\centering
	\includegraphics[scale=0.6]{img/LXe_n_lambda.pdf}
	\caption{\label{fig.nlambda} LXe refraction index as a function of the wavelength of the optical photon, as results from the parametrization in Ref.~\cite{Baldini:2004td}. For wavelengths in the emission range of TPB ($\sim$ 400--600 nm) the refraction index is practically flat, and has a value of 1.4.}
\end{figure}


Notice that, while the calculation of the term $\Delta d_g/c$~is immediate, finding the value
of $\Delta d_p/v_p$ requires the knowledge of the refraction index of the medium (which in turn is a function of the wavelength of the scintillation photons, see figure \ref{fig.nlambda}) and the interaction vertices of the gammas in the LXSCs, whose determination depends on the spatial resolution (and in particular of the DOI resolution) of the setup. Furthermore, the recorded time of the photoelectrons is affected by the time jitter of the sensor and the front-end electronics, as well as by the overall clock system. Other effects that affect the CRT are the existence of a rise constant for scintillation (absent in the case of LXe, but of the order of 90 ps for LYSO \cite{Seifert}), and the additional decay constant (about 2.2 ns) introduced if the SiPMs of the LXSC2 are coated with TPB. 

In the remaining of this section we study systematically the various factors contributing to the CRT in LXe. 

\subsection*{Intrinsic CRT: effect of the DOI resolution}
We start our study by assuming that the refraction indices in both media are constant, with values 1.7 for LXe (which is the average of the values it assumes in the scintillation wavelength range) and 
1.8 for LYSO \cite{nLYSO}. We also fix the refraction index of the SiPM window to the same value of the propagation medium (LXe or LYSO, depending on the setup), so that $n_1 = n_2$ and therefore all photons impinging the SiPM with an angle less than $90\,^{\circ}$~enter the sensor. These (unphysical) assumptions are made in order to separate the contributions to the CRT due to the scintillation lifetime and the DOI resolution from those related with the refraction index. 

\begin{figure}[!bhtp]
	\centering
	\includegraphics[scale=0.4]{img/FirstPEScintLXeLYSO.png}
	\caption{\label{fig.firstPE} Arrival time of the first recorded photon in LXe (blue line) and LYSO (green dotted), after subtracting the time of flight of the incident gamma ($d_g/c$) and the time of flight of the scintillation photons ($n \cdot d_p/c$). The presence of negative bins is due to the the sum of two effects: on one hand, the time registered by sensors is binned with a size of 5 ps, while the time of flight of incident gammas is taken with the precision provided by the Monte Carlo (sub picosecond). On the other hand, the time of flight of the scintillating photon is determined by the position of the SiPM which detects it. Since SiPMs have an area of 3 $\times$ 3 $\textrm{mm}^2$, the error associated is of the order of $\sim$ 1.5 mm, which translates to $\sim$ 5 ps for velocities similar to the one of light.}\label{fig.FirstPE}
\end{figure} 

Figure \ref{fig.FirstPE} shows the arrival time of the first photoelectron in LXe and LYSO, after subtracting the time of flight of the incident gamma ($d_g/c$) and the time of flight of the scintillation photons ($n \cdot x d_p/c$). As expected, given the faster decay time, the VUV photons emitted by LXe arrive to the sensors earlier (and with a narrower distribution) than the blue photons emitted by LYSO. 

Notice that $\Delta d_p$~is the difference 
between the position of the interaction vertex and the position of the SiPM which records the photoelectron. 
The plot is computed introducing a resolution of 2 mm FWHM both in the transverse coordinates and
in the DOI (which corresponds to the resolution expected in the LXSC2). The jitter introduced by the DOI correction is
$
%\begin{equation}
\delta \Delta t =\Delta z \times n \times 3.3\ \textrm{ps/mm}
%\label{eq.DOI}
%\end{equation}
$.
%
For LXe, with n = 1.7, $\delta \Delta t = 11.2$~ps, while for  LYSO, with n = 1.8, $\delta \Delta t = 11.9$~ps in both cases for a DOI resolution $\Delta z = 2$~mm. Notice, on the other hand, that a DOI resolution of 2 mm FWHM requires either a double layer of SiPMs (as in the LXSC2) or small monolithic crystals (for example, in reference \cite{VanDamm2011} the authors find a DOI resolution varying between 1 and 4.5 mm FWHM for LYSO crystals of 
$20 \times 20 \times 12 ~{\rm mm^3})$). Instead, a conventional segmented LYSO crystal, 20 mm
thick, would have a DOI resolution given by the digital resolution, which is determined by the pitch, $20{\textrm{ mm}}/\sqrt{12}$ = 13.3 mm, resulting in a $\delta \Delta t = 34$~ps.

\begin{figure}[!bhtp]
	\centering
	\includegraphics[scale=0.4]{img/DTOFLXeLYSO.png}
	\caption{\label{fig.dtof} $\delta \Delta t$ in LXe (blue line) and LYSO (green dotted line). The
	histograms are obtained introducing a spatial resolution of 2 mm for the transverse coordinates and
	the DOI, fixing the wavelength of the scintillation and the refraction index and setting the PDE of the SiPMs to one. }
\end{figure}

\begin{figure}[!bhtp]
	\centering
	\includegraphics[scale=0.4]{img/CRTvsPDELXeLYSONoJitterFstPE.png}
	\caption{\label{fig.crt1} CRT as a function of the PDE in LXe (blue line) and LYSO (green dotted line) computed using the first photoelectron. The calculation takes into account the scintillation yield and decay time as well as the DOI resolution but does not include
	the dependence of the refraction index in LXe with wavelength.}
\end{figure}

 Figure \ref{fig.dtof} shows $\delta \Delta t$ for LXe and LYSO for the ideal case of PDE = 1. The wider LYSO distribution is due to the slower decay constant.  Figure \ref{fig.crt1} shows the CRT as a function of the
 SiPM PDE. The CRT of the LXe setup for PDE = 1 is $\sim$11 ps (thus totally dominated by the uncertainty in DOI) while the intrinsic resolution of the LYSO setup for PDE = 1 is $\sim$30 ps, showing a large contribution from the slower decay constant. This values can be taken as the absolute best limit of the CRT in LXe and LYSO. For a PDE of 50\%, (best value of conventional SiPMs) the CRT in LYSO is $\sim$ 45 ps, while for a PDE of 20\% (best value of VUV--sensitive SiPMs) the CRT in LXe is $\sim$ 30 ps.   

\subsection*{Effect of the refraction index}
\begin{figure}[!bhtp]
	\centering
	\includegraphics[scale=0.6]{img/VelocityDistrLXe.pdf}
	\caption{\label{fig.vLXe} Distribution of the velocity of scintillation photons propagating in LXe, given the refraction index shown in figure \ref{fig.nlambda} and the emission spectrum shown in figure \ref{fig.spectrumLXe}.}
\end{figure}

The refraction index in LXe varies significantly with the scintillation photon wavelength, as shown in figure  \ref{fig.nlambda} (instead, the refraction index of LYSO is practically constant in the relevant range of scintillation
emission). This implies that the xenon VUV photons propagate with different velocities depending of their wavelength, as shown in figure \ref{fig.vLXe}. 
%To take into account this effect, one must use the average group velocity of the photons instead of the constant term $n/c$ in equation \ref{eq.CRT}. 
This introduces an extra smearing which contributes to the CRT, due to the spread of the average velocity. 

We also introduce the refraction index of the SiPM entrance window, which we take as $n_1 = 1.54$ (a typical value, quoted, for example for the SensL C-series 6 mm sensors \cite{nSensL}). The mismatch with the xenon refraction index
($n_2 \sim 1.7$ in the VUV region) or the LYSO refraction index ($n_2 = 1.8$) results in a reduction of the efficiency due to Fresnel reflections. The critical angle in xenon is $65\,^{\circ}$ ($58\,^{\circ}$~in LYSO) and the fraction of photons transmitted through the SiPM window is 63.5\% for LXe (56\% for LYSO).  

\begin{figure}[!bhtp]
	\centering
	\includegraphics[scale=0.40]{img/CRTvsPDE_phys_LXeLYSONoJitterFstPE.png}
	\caption{\label{fig.crt2} CRT as a function of the PDE in LXe (blue line) and LYSO (green dotted line) computed using the first photoelectron. The calculation takes into account the scintillation yield and decay time as well as the DOI resolution and the dependence of the refraction index in LXe with wavelength. The effect of Fresnel reflections
	due to the mismatch between the LXe (LYSO) refraction index and that of the SiPM is also taken into
	account.  }
\end{figure}

Figure \ref{fig.crt2} shows the CRT as a function of the
 SiPM PDE when the effect of the refraction index is taken into account. 
 The effect of the variable refraction index deteriorates considerably the intrinsic CRT in LXe, resulting in a value of $\sim$27 ps for PDE = 1 while it has little effect in LYSO. The CRT in LXe for a PDE of 20\% and in LYSO for a PDE of
 50\% is $\sim$~50 ps.
  
\subsection*{Effect of the SiPM  and front-end electronics jitter}
  Next we introduce the effect of the jitter of the SiPM and front-end electronics, smearing the arrival time of the photoelectrons by a gaussian distributions with $\sigma = \sigma_{\textrm{sipm}} +  \sigma_{\textrm{fee}}$ and setting 
 $\sigma_{\textrm{sipm}} = 80$~ps, $\sigma_{\textrm{fee}} = 30$~ps. These values are typical of modern SiPMs and front-end electronics \cite{FundamentalLimits}. 
 
 \begin{figure}[!bhtp]
	\centering
	\includegraphics[scale=0.40]{img/CTR_jitter_phys.png}
	\caption{\label{fig.crt3} CRT as a function of the PDE in LXe (blue line) and LYSO (green dotted line) computed using the first photoelectron. The calculation includes the effect of the SiPM  and front-end electronics jitter.}
\end{figure}

 \begin{figure}[!bhtp]
	\centering
	\includegraphics[scale=0.40]{img/lxe_noCher_avg_npe_phys.png}
	\caption{\label{fig.crt_avg_LXe} CRT as a function of the number of photoelectrons used to compute $\Delta t$, for LXe. The CRT is shown for several values of the PDE.}
\end{figure}

 \begin{figure}[!bhtp]
	\centering
	\includegraphics[scale=0.40]{img/lyso_noCher_avg_npe_phys.png}
	\caption{\label{fig.crt_avg_LYSO} CRT as a function of the number of photoelectrons used to compute $\Delta t$, for LYSO. The CRT is shown for several values of the PDE.}
\end{figure}

Figure \ref{fig.crt3} shows the CRT as a function of the
 SiPM nominal PDE for LX (red curve) and LYSO (green curve). The CRT 
for PDE = 1 is 90 ps for LXe and 100 ps for LYSO.  For a PDE of 50\% the CRT in LYSO is $\sim$ 125 ps, while for a PDE of 20\%  the CRT in LXe is $\sim$ 100 ps. However, the CRT can improve computing $\Delta t$~as an average of the first few photoelectrons, rather than using only the first photoelectron. This is shown in figures
\ref{fig.crt_avg_LXe} and \ref{fig.crt_avg_LYSO}. For a PDE of 20\%, it appears possible to reach a
CRT of 70 ps in the case of LXe, and about 85 ps in LYSO for a PDE of 50\%.

Interestingly, the results obtained for LYSO using the first photoelectron to compute $\Delta t$ (a CRT of 125 ps for a PDE of 50\%), agree rather well with the results of Ferri et al  \cite{LysoFBK} {\em when the LYSO crystals were cooled 
to $-20\,^{\circ}{\rm C}$}, e.g, reducing the DCR by a factor 16). The obvious advantage of LXe is the cryogenic operation, at $\sim -100\,^{\circ}{\rm C}$, which essentially eliminates the effect of the DCR. Thus, an excellent CRT of 70 ps could be achieved if the SiPM--VUV technology is capable of producing sensors with a PDE of 20\%. For a PDE of 15\% corresponding to the best existing devices, the CRT averaging over several photoelectrons is 80 ps, still very good.


\subsection*{Effect of the TPB decay constant}
%VUV sensitive SiPMs are currently still in the phase of development and are consequently very expensive (up to a factor 10) compared with blue sensitive SiPMs. 
% %We consider next the effect of coating the SiPMs of the LXSC2 with TPB. Notice that commercial VUV sensitive SiPMs are currently manufactured only by Hamamatsu\footnote{https://indico.hep.anl.gov/indico/getFile.py/access?contribId=157&sessionId=24&resId=0&materialId=slides&confId=625}, and suffer both from low PDE ($\sim 8\%$) and high cost (around 100 \euro\ per piece for 3 mm sensors, to be compared with 10-15 \euro\ for 3 mm conventional SiPMs). 
% Therefore VUV sensitive SiPMs are not a practical option
%for a PETALO scanner today, although the situation can improve considerably in the next few years given the interest in VUV sensitive sensors for the nEXO project \footnote{http://indico.inr.ru/event/4/session/1/contribution/8/material/slides/0.pdf} as well as for an upgrade of the 
%liquid xenon calorimeter of the MEG experiment \footnote{Eur.Phys.J. C73 (2013) no.4, 2365
%(2013-04-03)
%DOI: 10.1140/epjc/s10052-013-2365-2
%e-Print: arXiv:1303.2348}. 
%
%The baseline design of the LXSC2 proposes the use of conventional, blue-sensitive SiPMs coated with TPB, a technique that has been demonstrated by the NEXT experiment \cite{Alvarez:2012ub}. TPB coating introduces two effects in the response of a SiPM:
%
%\begin{enumerate}
%\item \item A reduction of the efficiency, associated with the absorption efficiency of the TPB at the LXe wavelengths ($\sim 80\%$) and the isotropic 
%re-emission of the blue photon (thus at most $\sim 50\%$~of the emitted photons enter the SiPM). 
% \item An effective delay of the first photoelectron due to the decay time of TPB ($\sim $ 1 ns). 
%\end{enumerate}

\begin{figure}[!bhtp]
	\centering
	\includegraphics[scale=0.4]{img/tpb_decays_pde_500ps.png}
	\caption{\label{fig.crtTPB} CRT as a function of the PDE in LXe (red line) and LXe coated  with TPB, assuming several decay times. The CRT is computed using the first photoelectron. The calculation takes into account the scintillation yield and decay time as well as the DOI resolution and the dependence of the refraction index in LXe with wavelength. }
\end{figure}

\begin{figure}[!bhtp]
	\centering
	\includegraphics[scale=0.4]{img/tpb_decays_jitter_500ps.png}
	\caption{\label{fig.crtTPBjit} CRT as a function of the PDE in LXe (red line) and LXe coated  with TPB, assuming several decay times. The CRT is computed using the first photoelectron. The effect of the jitter of the SiPM and electronics is included. }
\end{figure}

%\begin{figure}[!bhtp]
%	\centering
%	\includegraphics[scale=0.4]{../img/PetaloTOF/tpb_noCher_1d0ns_avg_npe_phys.png}
%	\caption{\label{fig.crtTPB1ns} CRT as a function of the number of photoelectrons used to compute $\Delta t$, for LXe and SiPMs coated with TPB. A decay constant of 1 ns is assumed. The CRT is shown for several values of the PDE. }
%\end{figure}
%
%\begin{figure}[!bhtp]
%	\centering
%	\includegraphics[scale=0.4]{../img/PetaloTOF/tpb_noCher_2d2ns_avg_npe_phys.png}
%	\caption{\label{fig.crtTPB2ns} CRT as a function of the number of photoelectrons used to compute $\Delta t$, for LXe and SiPMs coated with TPB. A decay constant of 2.2 ns is assumed. The CRT is shown for several values of the PDE. }
%\end{figure}
%
%\begin{figure}[!bhtp]
%	\centering
%	\includegraphics[scale=0.4]{../img/PetaloTOF/tpb_noCher_3d0ns_avg_npe_phys}
%	\caption{\label{fig.crtTPB3ns} CRT as a function of the number of photoelectrons used to compute $\Delta t$, for LXe and SiPMs coated with TPB. A decay constant of 3 ns is assumed. The CRT is shown for several values of the PDE. }
%\end{figure}


We now consider the case in which conventional SiPMs, with a PDE as large as 50\% in the blue region, are used in a PETALO scanner. The SiPMs would be coated with TPB to shift the VUV light to blue. The main reason to consider this scenario is cost. VUV sensitive SiPMs are currently much more expensive  than conventional SiPMs. It must be pointed out that such a cost difference can be quickly reduced with technological advances and massive production. Nevertheless, a PETALO scanner using TPB--coated SiPMs could be built with existing technology at a very competitive cost and is, therefore, interesting to study the effect of TPB on the CRT.

Figure \ref{fig.crtTPB} shows the CRT as a function of the
 SiPM nominal PDE for VUV sensitive SiPMs and TPB-coated SiPMs in LXe. 
 The calculation takes into account the scintillation yield and decay time as well as the DOI resolution and the dependence of the refraction index in LXe with wavelength, but not the jitter due to the SiPM and electronics. 
 Here we must compare the CRT achieved for VUV sensitive SiPMs with a PDE of 15--20\% ($\sim$ 50 ps) with the CRT achieved for blue sensitive SiPMs coated with TPB, taking as a reference a PDE of 50\% (this corresponds to the
 PDE for blue light,  however, the CRT is computed taking into account the effect of the TPB absorption and subsequent re-emission).  The CRT in this case varies from 140 ps (for a decay constant $\tau_{\rm{TPB}} = 1$~ns) to 
 200 ps (for a decay constant $\tau_{\rm{TPB}} = 3$~ns).  When the effect of the jitter of the SiPM and front-end electronics is added (figure \ref{fig.crtTPBjit}), the CRT (first photoelectron) for VUV-SiPMs is $\sim 100$~ps, while the CRT for TPB-coated SiPMs varies from  170 ps (for a decay constant $\tau_{\rm{TPB}} = 1$~ns) to  220 ps (for a decay constant $\tau_{\rm{TPB}} = 3$~ns). Averaging $\Delta t$ over the first 10 photoelectrons  the CRT is reduced to 130 ps for a decay constant $\tau_{\rm{TPB}} = 1$~ns;  155 ps for a decay constant $\tau_{\rm{TPB}} = 2.2$~ns;  and 170 ps for a decay constant $\tau_{\rm{TPB}} = 3$~ns. Notice that averaging over several photoelectrons reduces the error in the CRT by a large factor in the case of SiPMs coated with TPB. 
 
 %Averaging $\Delta t$ over the first 10 photoelectrons  the CRT is reduced to 130 ps for a decay constant $\tau_{\rm{TPB}} = 1$~ns (figure \ref{fig.crtTPB1ns});  155 ps for a decay constant $\tau_{\rm{TPB}} = 2.2$~ns (figure \ref{fig.crtTPB2ns});  and 170 ps for a decay constant $\tau_{\rm{TPB}} = 3$~ns (figure \ref{fig.crtTPB3ns}). Notice that averaging over several photoelectrons reduces the error in the CRT by a large factor in the case of SiPMs coated with TPB. 
 
 The results obtained with TPB are still very good. The best case scenario corresponds to a decay time of 1 ns
 (which can arguably be obtained with a thick enough coating) and yields a CRT of 130 ps, comparable to the value
 obtained by Ferri et al  \cite{LysoFBK}  when the LYSO crystals were cooled 
to $-20\,^{\circ}{\rm C}$.
    
%  \subsection*{Including the effect of the overall syncronization}
%  Finally, we have included the effect of the overall syncronization of a large system, introducing a binning of 25 ps in the recording of the photoelectrons (this corresponds to the binning offered by a number of TOF ASIC\footnote{PETSYS}). The effect turns out to be negligible.  
%  
%\subsection*{CRT of a PETALO-LXSC2 scanner}
%In summary, we find that the CRT of a PETALO-LXSC2 scanner can be as good as 35 ps rms or 80 ps FWHM if VUV sensitive SiPMs with a PDE of 15\% are used. This value seems to be within reach of the current technology, although high cost is still an issue for this type of devices. For reference, a PDE of 30\% would result in a CRT of 20 ps rms or 46 ps FWHM. 
%
%On the other hand, it appears feasible to build a large PETALO scanner using large (6 mm) and comparatively cheap SiPMs sensitive to blue light coated with TPB. For a typical PDE of 50\% we find a CRT of
%43 ps rms or $\sim$100 ps FWHM. A similar value is found for the LYSO reference setup.


\section{Summary and outlook}\label{sec.conclu}

We have studied the CRT that can be obtained by a PETALO scanner based in the LXSC2 cell. The intrinsic resolution of the cell (corresponding to an ideal VUV sensitive sensor of PDE one and forcing a fixed refraction index) is 12 ps, of which DOI uncertainty (2 mm) contributes with about 10 ps. Introducing a PDE of 15\%, corresponding to the best currently achieved by VUV-sensitive SiPMs, still results in an excellent CRT of  35 ps. Including the effect of the variable refraction index and Fresnel reflections, a CRT of 55 ps is found for a PDE of 15\%. Adding the effect of the jitter of the SiPM and electronics spoils the CRT to some 115 ps for a PDE of 15\%. Averaging $\Delta t$~over the first few photoelectrons, one recovers a CRT of 80 ps for a PDE of 15\% and 70 ps for a (technically feasible) PDE of 20\%. It follows that a PETALO scanner based in the LXSC2 cell equipped with VUV sensitive SiPMs of relatively large PDE (15\% or better) could reduce the CRT of a large system well below the 100 ps threshold, improving by a factor $\sim$ 4 the CRT of the best existing
PET-TOF commercial system. 

On the other hand, we find that using blue sensitive SiPMs coated with TPB results in a CRT that could be as good as
130--155 ps (for $\tau_{\rm{TPB}} = 1-2.2$~ns). These values still compete with the best results obtained for LYSO and would be, at least, a factor 2 better than those obtained with the VEREOS, at, arguably, much lower cost (the VEREOS uses LYSO crystals, and digital SiPMs, to be compared with much cheaper LXe and conventional, large-area SiPMs in the case of PETALO).  In summary, it appears feasible to build a PETALO scanner with existing technology, combining the advantages of large sensitivity, good energy and spatial resolution, and excellent CRT.  

\acknowledgments
The authors acknowledge support from the following agencies and institutions: the European Research Council (ERC) under the Advanced Grant 339787-NEXT; the Ministerio de Econom\'ia y Competitividad of Spain under grants CONSOLIDER-Ingenio 2010 CSD2008-0037 (CUP), FIS2014-53371-C04 and the Severo Ochoa Program SEV- 2014-0398; we acknowledge enlightening discussions with J. Varela and C. Lerche.

\bibliography{biblio}

\end{document}
